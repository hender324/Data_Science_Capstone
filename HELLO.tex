\documentclass[]{article}
\usepackage{lmodern}
\usepackage{amssymb,amsmath}
\usepackage{ifxetex,ifluatex}
\usepackage{fixltx2e} % provides \textsubscript
\ifnum 0\ifxetex 1\fi\ifluatex 1\fi=0 % if pdftex
  \usepackage[T1]{fontenc}
  \usepackage[utf8]{inputenc}
\else % if luatex or xelatex
  \ifxetex
    \usepackage{mathspec}
  \else
    \usepackage{fontspec}
  \fi
  \defaultfontfeatures{Ligatures=TeX,Scale=MatchLowercase}
\fi
% use upquote if available, for straight quotes in verbatim environments
\IfFileExists{upquote.sty}{\usepackage{upquote}}{}
% use microtype if available
\IfFileExists{microtype.sty}{%
\usepackage{microtype}
\UseMicrotypeSet[protrusion]{basicmath} % disable protrusion for tt fonts
}{}
\usepackage[margin=1in]{geometry}
\usepackage{hyperref}
\hypersetup{unicode=true,
            pdftitle={2029 U.S. Job Market Evaluation Proposal},
            pdfauthor={Roger Henderson},
            pdfborder={0 0 0},
            breaklinks=true}
\urlstyle{same}  % don't use monospace font for urls
\usepackage{graphicx,grffile}
\makeatletter
\def\maxwidth{\ifdim\Gin@nat@width>\linewidth\linewidth\else\Gin@nat@width\fi}
\def\maxheight{\ifdim\Gin@nat@height>\textheight\textheight\else\Gin@nat@height\fi}
\makeatother
% Scale images if necessary, so that they will not overflow the page
% margins by default, and it is still possible to overwrite the defaults
% using explicit options in \includegraphics[width, height, ...]{}
\setkeys{Gin}{width=\maxwidth,height=\maxheight,keepaspectratio}
\IfFileExists{parskip.sty}{%
\usepackage{parskip}
}{% else
\setlength{\parindent}{0pt}
\setlength{\parskip}{6pt plus 2pt minus 1pt}
}
\setlength{\emergencystretch}{3em}  % prevent overfull lines
\providecommand{\tightlist}{%
  \setlength{\itemsep}{0pt}\setlength{\parskip}{0pt}}
\setcounter{secnumdepth}{0}
% Redefines (sub)paragraphs to behave more like sections
\ifx\paragraph\undefined\else
\let\oldparagraph\paragraph
\renewcommand{\paragraph}[1]{\oldparagraph{#1}\mbox{}}
\fi
\ifx\subparagraph\undefined\else
\let\oldsubparagraph\subparagraph
\renewcommand{\subparagraph}[1]{\oldsubparagraph{#1}\mbox{}}
\fi

%%% Use protect on footnotes to avoid problems with footnotes in titles
\let\rmarkdownfootnote\footnote%
\def\footnote{\protect\rmarkdownfootnote}

%%% Change title format to be more compact
\usepackage{titling}

% Create subtitle command for use in maketitle
\newcommand{\subtitle}[1]{
  \posttitle{
    \begin{center}\large#1\end{center}
    }
}

\setlength{\droptitle}{-2em}

  \title{2029 U.S. Job Market Evaluation Proposal}
    \pretitle{\vspace{\droptitle}\centering\huge}
  \posttitle{\par}
    \author{Roger Henderson}
    \preauthor{\centering\large\emph}
  \postauthor{\par}
    \date{}
    \predate{}\postdate{}
  

\begin{document}
\maketitle

\begin{verbatim}
## [1] "Date:  October 16, 2018"
\end{verbatim}

\subsubsection{A.) Project Request:}\label{a.-project-request}

\begin{verbatim}
    Ten years from now, in 2029, there are likely to be significant changes in the make-up of the 
    United States workforce. Two unique questions come to mind:
    
    1.) Approximately what percentage of the United States workforce will be working in the gig 
        economy? The gig economy is defined as work that is temporary and flexible.
    2.) What percentages will be working remotely, in contract, freelance, or in some other form of a 
        non-traditional on-site position?
        
    The goal of this project is to gather an understanding of the existing market and trends to 
    forecast out employee makeup ten years into the future. The project is centered around core Amazon 
    business units; not the actions of subsidiaries.
    
\end{verbatim}

\subsubsection{B.) Stakeholder(s):}\label{b.-stakeholders}

\begin{verbatim}
    The primary client is Beth Galetti, Amazon's Senior Executive VP of Human Resources [7]. She is 
    primarily responsible for the planning of facilities and workspaces needed as part of the HQ2 
    expansion. The Executive VP wants to make effective use of resources and minimize expenses while 
    ensuring there are adequate workspaces available. Regardless of the Amazon timeline to begin 
    operations in the HQ2 facility, a long-term study is needed to effectively plan for the new 
    facility and the first few years of operational staffing. 

    The list of U.S. finalist cities as of the end of 2017 in alphabetical order are: Atlanta, Austin, 
    Boston, Chicago, Columbus, Dallas, Denver, Indianapolis, Los Angeles, Miami, Montgomery County 
    (PA), Maryland, Nashville, Newark, New York, Northern Virginia, Philadelphia, Pittsburgh, Raleigh, 
    and Washington D.C. Please note the shortlist is primarily centered around primarily east coast 
    cities [4].

    If considerably more talented workers will be found in the gig economy within the next ten years, 
    dramatic changes will need to be made in HR policies. Some of these changes include flexible       
    short-term compensation and benefits packages, an increase in wireless connectivity and security
    options and infrastructure to support remote operations, tax concerns related to free-lance     
    employees, the redefinition of a team and how to achieve team cohesion, etc.

    Decisions will need to be made at geographical, work classification (gig, seasonal temporary, 
    contract, full-time permanent), and department (fulfillment center, driver, administrative, 
    software development, analytics) levels.

    Per Amazon's 2017 Annual Report on the Investor Relations site, as of December 31, 2017, the 
    company employed approximately 566K full-time and part-time employees. In 2017 alone, Amazon 
    directly created more than 130K new permanent, full-time Amazon jobs, not including acquisitions, 
    bringing its permanent global employee base to just over 560K [1].

    The Company also utilizes independent contractors and temporary personnel to supplement the 
    workforce in three main channels. The first channel is utilizing temporary seasonal workers in its 
    warehouses and customer service centers. According to CNET and Fortune Magazine, utilizing 2016 
    data, Amazon traditionally hires between 70K to 120K seasonal employees, with more than 14K of 
    those positions converting to regular, full-time roles [5] [13]. Considerations for seasonal 
    workers center on providing equitable contracts and ensuring available work center capacity.

    There is limited information available to the public however it is important to recognize that most
    of Amazon's non-traditional workers are temporary seasonal workers and the number of season workers
    increases 5K to 10K each year [5]. Fulfillment and customer care centers can currently be found in 
    the U.S. regions listed below.
\end{verbatim}

\begin{verbatim}
##         region                                                state
## 1 New England:  Connecticut, Delaware, Massachusetts, New Hampshire
## 2    Atlantic: Maryland, New Jersey, New York, North Carolina, etc.
## 3   Southeast:            Alabama, Florida, Georgia, South Carolina
## 4     Central:    Illinois, Indiana, Kentucky, Michigan, Ohio, etc.
## 5     Midwest:                          Kansas, Minnesota, Oklahoma
## 6   Southwest:                                                Texas
## 7    Mountain:                      Arizona, Colorado, Nevada, Utah
## 8     Pacific:                       California, Oregon, Washington
\end{verbatim}

\begin{verbatim}
    The second channel of temporary workers is Amazon Flex gig drivers who deliver Amazon Prime 
    packages to customers. Glassdoor estimates the number of these flexible employees are between 
    5K to 10K employees [6]. Considerations for gig workers center on effective training and 
    clearly defined roles to ensure all local, state, and federal tax and gig employment guidelines 
    are being followed. Amazon would not wish to incorrectly classify contract workers as gig employees.

    Amazon Flex is located across the U.S with operations located in New England, Atlantic, Southeast, 
    Central, Midwest, Southwest, Mountain, and Pacific regions. Below is the list of cities in which 
    Amazon Flex is offered with the corresponding U.S. census geographical region [15]. 
\end{verbatim}

\begin{verbatim}
##         region                                           state
## 1 New England:                                          Boston
## 2    Atlantic: New Jersey, Philadelphia, Pittsburgh, Charlotte
## 3   Southeast:                           Miami, Orlando, Tampa
## 4     Central:                 Detroit, Milwaukee, Minneapolis
## 5     Midwest:           Kansas City, St. Louis, Kentucky Area
## 6   Southwest:                                          Dallas
## 7    Mountain:                                  Denver, Tucson
## 8     Pacific:        Pacific - Greater Los Angeles Area, etc.
\end{verbatim}

\begin{verbatim}
    As this is a relatively new service, there is not enough data currently available to estimate Amazon 
    Flex specific growth trends. However, we can surmise that the service is likely to expand further 
    because after Amazon's acquisition of Whole Foods in August 2017, Amazon Flex has steadily expanded 
    into new locations [12].
    
    The third channel of temporary workers is personnel in the corporate office who are employed through 
    staffing agencies. Contractors working at Seattle, Washington headquarters locations are comprised of 
    the following occupations and approved contract agencies:
    
\end{verbatim}

\begin{verbatim}
##              job_functions                              staffing_agencies
## 1      IT, Web, & Software      OSI Engineering, Prime 8 Consulting, etc.
## 2     Finance & Accounting   Affirma Consulting, Robert Half, Rylem, etc.
## 3 Administrative Functions     Aerotek, Concentrix Daksh, Corestaff, etc.
## 4        Design & Creative Creative Circle, Filter, Simplicity Consulting
## 5            Manufacturing    Aerotek, Integrity Staffing Solutions, etc.
## 6        Sales & Marketing                  Kforce, Simplicity Consulting
## 7          HR & Recruiting                   Rylem, Simplicity Consulting
## 8              Engineering                                OSI Engineering
\end{verbatim}

\begin{verbatim}
    Amazon does not release the number of corporate contract employees. The company has U.S. software 
    development centers in the locations listed below [2-3, 8-11, 14].      
    
            
\end{verbatim}

\begin{verbatim}
##         region                                                       city
## 1 New England:                                                  Cambridge
## 2    Atlantic:                         New York, Pittsburgh, Herndon (VA)
## 3     Midwest:                                                Minneapolis
## 4   Southwest:                                                     Austin
## 5    Mountain:                                                     Temple
## 6     Pacific: Cupertino, Irvine, San Francisco, San Luis Obispo, Seattle
\end{verbatim}

\begin{verbatim}
    Amazon may wish to recruit more contract employees because there is tenure flexibility based on 
    production needs and the company can reduce risk by getting an opportunity to try out an employee 
    to determine if they are a good fit before committing to hiring them. 

    Staffing agencies usually help organizations cut expensive hiring costs and create a candidate 
    profile that engender better job candidates. There are legal issues to consider when considering 
    staffing company employees including the possible misclassification of employees for tax purposes, 
    compliance with labor law and federal laws, and possible contract disputes. 

    In addition to the considerations that arise with gig and contract workers, if more permanent future 
    employees expect to have partial or full remote working capability in HQ2 facilities this has a 
    significant impact on the extent of the construction. 

    If more employees will wish to work remotely, Amazon will want to accommodate the needs of the workforce 
    and can save considerable expense in construction, rent, and utilities. Amazon can utilize the findings 
    of this analysis to better decide how many facilities and the dimensions of the facilities needed to 
    accommodate its onsite HQ2 personnel whether they be contract, gig, or permanent. 

    In addition to the construction and ongoing expenses related to urban workspaces, Amazon may wish to lead 
    the effort to employ more gig and remote workers, to obtain other benefits. These benefits include access 
    to a bigger talent pool not limited by geographical location, tapping into less inflated high salary job 
    markets, and accessing remote workers who are generally happier and more productive. 

    If there is a significant trend to more flexible working options, HR will want to begin long-range planning. 
    The planning could include hiring and training leaders that can effectively manage virtually with the right 
    balance of autonomy and accountability, determining the best options for setting up high capacity remote 
    systems connectivity, legal employee contact changes, and several other considerations.
\end{verbatim}

\subsubsection{C.) Data Sources:}\label{c.-data-sources}

\begin{verbatim}
   . Department of Labor: Closely watched measures of employment and unemployment. https://www.dol.gov/
   
   . Employment by U.S. Census: Data that measures the state of the nation's workforce, including employment
     and unemployment levels, as well as weeks and hours worked. https://www.census.gov/topics/employment.html
     
   . US Government Open Data: United States open source software government site making data open and 
     machine-readable on a variety of topics. https://www.data.gov/open-gov/
     
   . Bureau of Labor Statistics: U.S. government's data collection of employment-related stats across 
     regions, states, and local areas. 
     https://www.bls.gov/opub/mlr/2018/article/electronically-mediated-work-new-questions-in-the-contingent
     -worker-supplement.htm
     
   . JP Morgan Chase Online Platform Economy in 2018: JP Morgan Chase's evaluation of the global economy 
     which provides gig economy findings https://www.jpmorganchase.com/corporate/institute/document/
     institute-ope-2018.pdf
     
   . MIT Technology Review: MIT online article that addresses how the US Bureau of Labor Statistics is 
     likely undercounting alternative work arrangement employees.      
     https://www.technologyreview.com/s/611381/the-us-government-is-seriously-underestimating-how-much-
     americans-rely-on-gig-work/
\end{verbatim}

\subsubsection{D.) Approach:}\label{d.-approach}

\begin{verbatim}
    The general approach to answering the 2029 United States gig and remote workers percentages questions 
    are listed below.

    1. Obtain data on current gig and remote workers makeup. The goal is to gain a U.S employee 
       multi-dimensional data set that includes the job classification, demographic factors, 
       geographic factors, economic factors, plus the fraction of gig/remote workers.  
    2. Gain an understanding if the trend is increasing or decreasing and by what percentages 
       in the U.S.
        a.  Employ the statistical classification model to classify current employment data points 
            into increases or decreases in gig and remote employment. 
        b.  Employ the clustering classification model to group like employment data points into 
            increases or decreases in gig and remote employment clustered by geographical region. 
        c.  Employ multiple regression analysis models to estimate the relationship between employment 
            data points to determine an upward or downward trend in gig and remote employment.
    3. If all three models (classification, clustering, and regression show a consistent upward or 
       downward trend) gain an understanding of the factors behind the trend.
    4. Adjust the data points, using the factors behind the trend, to obtain new 2029 data points. 
    5. Rank the potential U.S. future employment scenarios using cluster analysis.
    6. Apply trend analysis to update the estimates to get the final estimated impact by job type on 
       the Amazon HQ2 hiring. This is the step where the analysis moves from U.S. analysis and become 
       to become Amazon.com specific.
    7. Provide Amazon.com estimates of permanent, gig, and contract 2029 employees by occupations and 
       geographical level.
    8. Build visualizations to illustrate the differences between both U.S. and Amazon.com employee 
       data in 2019 and 2029 predicted data by geographic locations and occupations for proposed HQ2 
       locations.
       
\end{verbatim}

\subsubsection{E.) Deliverables:}\label{e.-deliverables}

\begin{verbatim}
    The deliverables are below:
    a.) Slide deck with the following:
        1.  an understanding of the project.
        2.  a summary of the original data points.
        3.  classification, clustering, and regression analysis that produce predicted data points.
        4.  estimated 2029 data points alongside 2019 data points with percentage differences.
    b.) Any programming and data munging code.
    c.) All data in a cleaned form that was used for analysis. 
    d.) Trained usable models that can be used to run future data through using another factor 
        (i.e. GDP growth).

        When the project is complete, the project can be listed on Data Science Central and LinkedIn 
        to illustrate competency.
\end{verbatim}

\subsubsection{F.) Works Cited:}\label{f.-works-cited}

\begin{verbatim}
    1.) Amazon blog, The. Investor Relations. (2018, October 13). Retrieved from https://ir.aboutamazon.com/.
    2.) "Amazon expanding Detroit presence". Archived from the original on 2016-07-11. 
        Retrieved October 14, 2018.
    3.)  "Amazon Minneapolis". Archived from the original on 2016-06-10. 
    4.) Bisnow Team. Forbes. "Amazon HQ2 Shortlist: Details on the 20 Finalists In $5B Sweepstakes." 
        (2018, October 14). Retrieved from 
        https://www.forbes.com/sites/bisnow/2018/01/19/amazon-hq2-shortlist-details-on-the-20-finalists-
        in-5b-sweepstakes/. 
   5.)  Carson, E. (2017). "Amazon will hire 120,000 temporary workers this holiday season." CNET. Retrieved 
        from https://www.cnet.com/news/amazon-plans-to-hire-120000-temporary-workers-this-holiday-season/. 
   6.)  Glassdoor. Amazon Flex Overview. (2018, October 14). Retrieved from 
        https://www.glassdoor.com/Overview/Working-at-Amazon-Flex-EI_IE1324363.11,22.htm.
   7.)  Kim, E. (2017). "Here's who really runs Amazon - and only 2 of the top 38 execs are women." Retrieved 
        from https://www.cnbc.com/2017/11/28/who-are-amazons-top-executives.html. 
   8.)  Kirsner, Scott (December 23, 2011). "Amazon plans Cambridge office". Boston Globe. Archived from the 
        original on January 16, 2013. Retrieved October 14, 2018. 
   9.)  Neibauer, Michael. "Amazon's Herndon employees will earn $114K on average". Archived from the original 
        on June 19, 2015. Retrieved October 14, 2018. 
  10.)  Novak, Shonda (November 12, 2014). "Sources: Amazon.com to bring 200-plus tech jobs to Austin". 
        Austin-American Statesman. Archived from the original on February 26, 2015. Retrieved October 14, 2018. 
  11.)  "Pittsburgh". Archived from the original on January 17, 2017. Retrieved October 14, 2018. 
  12.)  Redman, R. (2018). "Growth spurt for Prime Now grocery delivery." SN. Supermarket News. (2018, October 
        14). 
        Retrieved from https://www.supermarketnews.com/online-retail/growth-spurt-prime-now-grocery-delivery. 
  13.)  Reuters. Fortune. (2018, October 14). Retrieved from 
        http://fortune.com/2016/10/13/amazon-120000-temporary-workers-holiday-season/.
  14.)  "San Luis Obispo". a2z.com. Archived from the original on March 3, 2012. Retrieved October 14, 2018. 
  15.)  Side Husl. Amazon Flex. (2018, October 14). Retrieved from https://sidehusl.com/amazon-flex/. 
\end{verbatim}


\end{document}
